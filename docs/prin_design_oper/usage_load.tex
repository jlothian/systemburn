\subsection{Load Files}

The load file for SystemBurn has been designed with flexibility in
mind. During execution, a single load is run at a time, with the same load
running on each MPI process. Each load contains a run time value and is
also subdivided into one or more sub-loads, each of which contain some
number of plans. Each sub-load is also associated with a CPU set mask,
some subset of the CPU cores available to the MPI process running the
load. With this association, plans contained in a sub-load with a given
CPU set can be required to run on only the CPUs in that set. An example
use of this capability is running SystemBurn on a dual socket system. Two
sub-loads could be defined, with CPU sets that placed each sub-load on
its own socket. Then, a third sub-load could have a CPU set encompassing
both sockets. These CPU sets can be managed at two levels: individually,
with custom masks for each sub-load; or at the load level, by dividing
the available CPU cores up based on the number of sub-loads and a user
specified pattern (block or round robin). More detailed description of
these load parameters are given below.

The load file allows the user to set the following general parameters
for each load:

\begin{description}
	\item [Run Time] This allows the user to specify how long they would like an individual load to run for.
	\item [CPU Set Type] This parameter has three different options: 
	\begin{enumerate}
		\item Block: This sets the CPU masks such that each
		sub-load is assigned a set of adjacent CPU cores. For
		example: A node with 12 CPUs and 3 CPU sets will have
		an affinity mask which looks as follows:

\begin{verbatim}
SET     CPU(s)
 0      0  1  2  3
 1      4  5  6  7
 2      8  9 10 11
\end{verbatim}

		\item Round Robin: This sets the affinities in the
		classic round robin distribution. For example: A node
		with 12 CPUs and 3 CPU sets will have an affinity mask
		which looks as follows:

\begin{verbatim}
SET     CPU(s)
 0      0  3  6  9
 1      1  4  7 10
 2      2  5  8 11
\end{verbatim}

		\item Sub-load Specific: This option allows the user to
		specify exactly which cores are to be put in each CPU
		set. This allows for a very low level of control when it
		comes to which CPUs run a specific set of modules. To use
		this affinity type the individual masks for each CPU set
		must be specified within the associated sub-load. This
		can be used to set more unbalanced loads:

\begin{verbatim}
SET     CPU(s)
 0      8 9
 1      1 2 3 4 5 6 7
 2      0
\end{verbatim}

	\end{enumerate}
\end{description}

In addition to the general parameters that apply to the entire load,
the load can also contain any number of sub-loads, each possessing
several other pieces of information associated with that sub-load:

\begin{description}
	\item [Copies] A single number in each sub-load determines how
	many distinct copies of that sub-load should be run as part of
	running the full load. This allows the file to be more concise
	when repetition occurs in the load. 
	\item [CPU Set Mask] This is a list of CPU cores that should be
	included in the sub-load's CPU set (if the load's CPU set type
	is sub-load specific). If there are any copies of the load,
	a CPU set mask should be specified for each copy.
\end{description}

Also, each sub-load will contain one or more plans, each of which has its own parameters:

\begin{description}
	\item [Copies] For the same reason that sub-loads can specify
	copies, a single parameter in a plan specifies the number of
	copies of that plan are in the sub-load.
	\item [Load Module] Ultimately, most of SystemBurn's work is
	performed by load modules (see the next section, Modules). Each
	plan contains the name of a single module for it to run.
	\item [Load Module Parameters] Each load module takes some kind
	of input parameter(s). Thus, each plan requires values that
	serve as the input parameters for the specific module used.
	Note: plans accept a parameter for the amount of memory the 
	plan should use. This can be specified by the size followed by
	the byte prefix; i.e. MB, KB, GB ... ; m,k,g,p,t, and e work as well.

\end{description}

The amount of customization available within the load files should allow
SystemBurn to emulate a vast array of different application stresses. The
run time parameter can be altered to run several different modules
in a short amount of time, or a few modules for a very long period of
time. The number of CPU sets and the different affinity types allow the
user to bind threads to specific CPUs or to have unbound threads float
over the entire node. A combination of these 2 extremes is also possible.

The syntax of the load files is based on a collection of keywords,
each specifying a certain function. Each keyword is then followed by
the necessary arguments (if any) on the same line.

\begin{description}
	\item [LOAD\_START] This keyword signals the beginning of a load entry within the file. It can be replaced with an opening brace: \verb!"{"!.
	\item [RUNTIME] Specifies a run time value (in seconds) for the load as an integer immediately following the keyword.
	\item [SCHEDULE] Specifies a CPU set type from the three options described above, using these keywords: \verb!BLOCK!, \verb!ROUND_ROBIN!, or \verb!SUBLOAD_SPECIFIC!.
	\item [SUBLOAD] Acts as a header to a sub-load, indicates the number of copies of the next sub-load as an integer following the keyword.
	\item [SUB\_START] Begins a sub-load entry, within a load entry. It can be replaced with an opening square bracket: \verb!"["!.
	\item [PLAN] Starts a plan entry. Immediately followed by the number of copies, the module name, and all module parameters.
	\item [MASK] Specifies a CPU set, and is followed by any number of integer core numbers to be associated with the current sub-load. This keyword is only used when \verb!SUBLOAD_SPECIFIC! is the CPU set type. Also, If a sub-load has multiple copies, the \verb!MASK! keyword must be used multiple times: one for each copy.
	\item [SUB\_END] Ends a sub-load entry. It can be replaced with a closing square bracket: \verb!"]"!.
	\item [LOAD\_END] Signals the end of a load entry, and can be replaced with a closing brace: \verb!"}"!.
\end{description}

Below is a template for a load file used by SystemBurn. 

\begin{verbatim}
LOAD_START
        # GENERAL INFORMATION
        RUNTIME <# of seconds>

        # Affinity Type
        SCHEDULE <BLOCK, ROUND_ROBIN, or SUBLOAD_SPECIFIC>

        # SUBLOADS (Specify different loads for different cpu sets)

        SUBLOAD <# of copies>
        # Number of cpu sets to run this subload on
        SUB_START
                PLAN <# of copies> <module> <module parameters...>
                ...

                MASK <cpu cores...>
        SUB_END

        ...

LOAD_END
\end{verbatim}

Example Load File 1:
\begin{verbatim}
LOAD_START
        # 16 core system
        RUNTIME 60
        SCHEDULE SUBLOAD_SPECIFIC

        SUBLOAD 3
        SUB_START
                PLAN 2 GUPS 25m
                PLAN 1 DSTRIDE 1000000m
                PLAN 1 FFT2D 3000m

                MASK 0 2 4 6
                MASK 1 3 5 7
                MASK 8 9 10 11
        SUB_END

        SUBLOAD 1
        SUB_START
                PLAN 2 FFT1D  1000000Mb
                PLAN 1 GUPS 24M
                PLAN 1 GUPS 26m

                MASK 12 13 14 15
        SUB_END
LOAD_END
\end{verbatim}

Example Load File 2:
\begin{verbatim}
{
        # simple load for a four socket system with 8 cores per socket
        # each socket runs 4 smaller PV1 loads and 4 larger PV2 loads
        RUNTIME 450
        SCHEDULE BLOCK
        SUBLOAD 4
        [
                PLAN 4 PV1  800m
                PLAN 4 PV2 1600m
        ]
}
\end{verbatim}


