\subsection{Configuration Files}

The configuration file, \verb!systemburn.config!, contains several
important values which control the SystemBurn suite's overall behavior. A
note on syntax: all characters on a line after a \verb!#! symbol are
treated as comments, and each parameter is associated with a specific
keyword that must precede the value on that line; if one of the keywords
is not used, a default value will be substituted.

\begin{description}

	\item[Maximum Number of Worker Threads] Keyword:
	\verb!WORKERS!. Default: 16 threads. This value specifies the
	maximum number of execution threads per MPI process to create
	when scheduling the current load. If the number of module loads
	in the load file exceeds the number given here, then the entire
	module load will not be scheduled, but only the first modules
	up to this maximum number.

	\item[Thermal Relaxation Time] Keyword:
	\verb!REST_TIME!. Default: 10 seconds.
	This is the time period for which the code will idle before
	beginning the set of loads and then again after completing the
	set of loads. The intent is to allow the idling system to 
	come into a baseline thermal equilibrium before and after the 
	loads. It should be noted that the loads should also run long
	enough for the system to come into a stressed thermal equilibrium
	for an accurate max temperature.
	This value can be set to 1 for quick testing.
	For small systems which have little impact on their environment
	values between 30-300 seconds are generally sufficient.
	Recommendations for values for larger systems which do
	affect their environment have yet to be developed but will likely
	run much longer to allow machine room power and cooling to also
	reach equilibrium.

	\item[Emergency Shutdown Temperature] Keyword: \verb!MAX_TEMP!. Default:
	70 degrees Celsius. This value specifies the maximum temperature
	a node can reach before SystemBurn shuts itself down. This feature is 
	only enabled on newer kernels which support temperature monitoring.

	\item[Monitor Thread Checking Period] Keyword:
	\verb!MONITOR_FREQ!. Default: 1 second. This specifies how often
	the monitor thread retrieves temperature data from the system.
	This feature is only enabled on newer kernels which support
	temperature monitoring.

	\item[Monitor Thread Output Period] Keyword:
	\verb!MONITOR_OUT!. Default: 10 seconds. This specifies how
	often the monitor thread will print its temperature information,
	if output from threads other than the main thread have been
	allowed (see the verbosity command line option, above).
	This feature is only enabled on newer kernels which support
	temperature monitoring.

\end{description}

An example configuration file appears as follows: 

\begin{verbatim}

# systemburn.config

# This file contains the default configuration
# information for systemburn. Comments are
# denoted by a '#' sign, and the needed numbers
# are described by the comments below:

# Number of worker threads:
WORKERS 16

# Thermal Relaxation time: (seconds)
# This is the time period for which the code will idle before
# beginning the set of loads and then again after completing the
# set of loads. The intent is to allow the idling system to 
# come into a baseline thermal equilibrium before and after the 
# loads. It should be noted that the loads should also run long
# enough for the system to come into a stressed thermal equilibrium
# for an accurate max temperature.
# This value can be set to 1 for quick testing.
# For small systems which have little impact on their environment
# values between 30-300 seconds are generally sufficient.
# Recommendations for values for larger systems which do
# affect their environment have yet to be developed but will likely
# run much longer to allow machine room power and cooling to also
# reach equilibrium.
REST_TIME 10

# The following are only useful on newer systems that support
# internal temperature monitoring via the kernel's /sys and
# /proc interfaces. Generally, this requires that the appropriate
# Linux kernel module (k8temp, k10temp, coretemp) or equivalent
# be loaded into the kernel.

# Emergency Shutdown Temperature: (degrees Centigrade)
# Over Temp protection requires access to the processor temperature.
# A few important notes:
# on AMD systems (esp k8) the 
MAX_TEMP 85

# Temperature Monitoring Frequency (in seconds)
# This determines how often the internal core temperature data is
# checked/updated and how often we check for an overtemp emergency.
MONITOR_FREQ 5

# Monitor thread ouput period:
# This is how often we summarize and print the temperatures.
# Note that the summary requires an MPI_REDUCE or equivalent
# and can thus be expense on larger system partitions. Adjust
# the output frequency accordingly:
# Small systems: 15-60 seconds
# larger systems: 
MONITOR_OUT 15

\end{verbatim}
